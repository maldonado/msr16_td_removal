% -*- root: main.tex -*-
The main goal of our study is to better understand what happens to \SATD once it is introduced into software projects. To do so, our first step is to quantify how much of \SATD comments gets removed (RQ1). Then, we analyze if the developers have sense of ownership with they \SATD comments by analyzing who are the authors of the changes that introduced and then removed these \SATD comments (RQ2). 

\vspace{3mm}
\noindent\rqi
\vspace{3mm}

\noindent \textbf{Motivation:} Previous work showed that technical debt is widespread, unavoidable and have a negative impact in the quality of software projects~\cite{Lim2012Software}. Therefore, intuitively we expect that removing technical debt from the source code is a concern to the developers. To better understand how developers deal with technical debt we must first identify and quantify how much debt is introduced and removed. Also, answering this question will provide us insight if source code comments indeed helps developers to manage technical debt. 

\vspace{1mm}
\noindent \textbf{Approach:} As described in subsection \ref{sub:checkout_all_versions_of_files} we have stored all different versions of all source code files. For each analyzed \SATD comment we take the oldest file version available on which the debt was found and incrementally search for matches in all latter versions of this file. The first time that the analyzed \SATD comments appears in a file points out to the exact file version that the \SATD comment was introduced in the system.

To analyze if the introduced \SATD comment was removed we keep searching for matches of the comment in the remaining file versions available. Therefore, the first file version that we are not able to find a match for the analyzed comment is also the version that the \SATD comment was removed. Another possible way to remove a \SATD comment is by deleting the file that contains it. 

\vspace{1mm}
\noindent \textbf{Results:} 



\conclusionbox{Avg. 73\%. Range: 40-90\% of TD gets removed.}

\vspace{3mm}
\noindent\rqii
\vspace{3mm}

\noindent \textbf{Motivation:} 
Intuition tells us that it would be more difficult to have other address TD since 1) they need to know how to address the issue and 2) they may not know where the TD is.

\vspace{1mm}
\noindent \textbf{Approach:} 
Looked at the commit authors to see if the remover is the same as the introducer.

\vspace{1mm}
\noindent \textbf{Results:} 
\begin{itemize}
\item Add table/figure showing the amount that is self-removed.
\end{itemize}


\conclusionbox{Avg. 54\% is self-removed. Range 25-71\%.}

\vspace{3mm}
\noindent\rqiii
\vspace{3mm}

\noindent \textbf{Motivation:} 
We would like to understand how long TD survives so that projects can know if it is typical to have TD in their project for that long or not.
\vspace{1mm}
\noindent \textbf{Approach:} 
We examine the time between introduction and removal.

\vspace{1mm}
\noindent \textbf{Results:} 
\begin{itemize}
\item Add figures of distribution of removal time.
\item Add survival plots
\item file addition vs. removal
\item Difference between self vs. non-self removal.
\end{itemize}

\conclusionbox{Avg. median is 30 days.}

 \vspace{3mm}
 \noindent\rqiv
 \vspace{3mm}

 \noindent \textbf{Motivation:} 
We want to know what activities tend to lead to the removal of TD.
 \vspace{1mm}
 \noindent \textbf{Approach:} 
Examine the commits messages of the removal commits and run topic models.
 \vspace{1mm}
 
 \noindent \textbf{Results:} 
\begin{itemize}
\item Add topics/activities that lead to the removal.
\end{itemize}
 \conclusionbox{}
% -*- root: main.tex -*-
% !TEX root = main.tex
Self-admitted technical debt refers to technical debt that can be detected through code comments. Prior work examined the detection, management and impact of \SATD. However, little is known about the removal of such technical debt. In this paper, we conduct an empirical study to examine how much \SATD is removed, how long such technical debt lives in a project before removal and who removes such debt. We find that the majority of \SATD is removed (74.4\% on average), that \SATD is mostly self-removed (54.4\% on average), and that it lasts between 82 - 613.2 days on average in a project before it is removed. Then, we conduct a survey with 14 developers to understand the reasons for the introduction and removal of \SATD. We find that there is no formal process to remove \SATD, and most removals occur as part of bug fixing.

Our results provide insights that indicate that \SATD is important, which is why the majority of it is removed. Also, they suggest that although developers are aware of the need to remove \SATD, most projects do not employ any formal process to address it. Hence, techniques are needed to allow projects to effectively and systematically address \SATD.

In the future, we plan to perform qualitative studies that examine the `whys' of our findings. In particular, we would like to examine why developers tend to self-remove technical debt. Additionally, we plan to better understand why some projects remove less \SATD than others. Finally, we plan to study the introduction and removal of \SATD of the projects at the revision level since that may provide a deeper understanding of the removal of \SATD.


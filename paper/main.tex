% ----------------------------------------------------------------------------------------------------------------
% For tracking purposes - this is V2.0 - May 2012

\documentclass[10pt, conference]{IEEEtran}


\usepackage{amssymb,amsmath}
\usepackage{wrapfig}
\usepackage{multirow}
\usepackage{makecell}
\usepackage{graphicx}
\usepackage{algorithm}
\usepackage{algorithmic}
%\usepackage{times}
\usepackage{cite}
\usepackage{url}
\usepackage{booktabs}
%\usepackage{subfigure}
\usepackage{fancybox}
\usepackage{color}
\usepackage{array}
%\usepackage{subfigure}
\usepackage{caption,subcaption}
\usepackage{balance}
\usepackage{epstopdf}
\usepackage{array}
\usepackage{xspace}







\graphicspath{ {../images/} }

% author comments with colors 
\newcommand{\emad}[1]{\textcolor{red}{{\it [Emad: #1]}}}
\newcommand{\alexander}[1]{\textcolor{red}{{\it [Alexander: #1]}}}
\newcommand{\everton}[1]{\textcolor{blue}{{\it [Everton: #1]}}}
\newcommand{\rabe}[1]{\textcolor{blue}{{\it [Rabe: #1]}}}
\newcommand{\todo}[1]{\colorbox{yellow}{\textbf{[#1]}}}

% conclusion box for summarize the research questions
\newcommand{\conclusionbox}[1]{%
       \vspace{2mm}
       \framebox[0.45\textwidth][c]{%
              \parbox[b]{0.42\textwidth}{%
                     {\it #1}
              }
       }
       \vspace{2mm}
}


\newcommand{\conclusion}[1]{%
\begin{center}
	\noindent\thicklines\setlength{\fboxsep}{8pt}\cornersize{0.03}\Ovalbox{\begin{minipage}{3in}\textit{
	\textbf{#1}
}\end{minipage} }
\end{center}
}











\newcommand{\rqi}{\textbf{RQ1. How much self-admitted technical debt gets removed?\\}}
\newcommand{\rqii}{\textbf{RQ2. Who removes self-admitted technical debt? Is it most likely to be self-removed or removed by others?\\}}
\newcommand{\rqiii}{\textbf{RQ3. How long does self-admitted technical debt survive in a project?\\}}
% \newcommand{\rqiii}{\textbf{RQ3. How long does self-admitted technical debt survive in a project?\\}}
\newcommand{\rqiv}{\textbf{RQ4. What activities lead to the removal of self-admitted technical debt?\\}}

\newcommand{\SATD}{self-admitted technical debt\xspace}
\newcommand{\CSATD}{Self-admitted technical debt\xspace}



\begin{document}

\title{An Empirical Study On the Removal of Self-Admitted Technical Debt}
\author{
%%	\alignauthor 
%	Everton da S. Maldonado\textsuperscript{\S}, Rabe Abdalkareem\textsuperscript{\S}, Emad Shihab\textsuperscript{\S}, Alexander Serebrenik\textsuperscript{\textdagger}\\
%	\affaddr{}
%	\affaddr{\textsuperscript{\S}Dept. Computer Science and Software Engineering, Concordia University, Montreal, Canada}\\
%	% \email{e\_silvam@encs.concordia.ca}
%	% \email{eshihab@cse.concordia.ca}
%%	\alignauthor 
%%	Alexander Serebrenik\\ 
%	\affaddr{\textsuperscript{\textdagger}Eindhoven University of Technology, Eindhoven, The Netherlands}\\
%	\email{\url{{e_silvam,rab_abdu,eshihab}@encs.concordia.ca}, \url{a.serebrenik@tue.nl}}
}
%\author{\IEEEauthorblockN{Authors Name/s per 1st Affiliation (Author)}
%\IEEEauthorblockA{line 1 (of Affiliation): dept. name of organization\\
%line 2: name of organization, acronyms acceptable\\
%line 3: City, Country\\
%line 4: Email: name@xyz.com}
%\and
%\IEEEauthorblockN{Authors Name/s per 2nd Affiliation (Author)}
%\IEEEauthorblockA{line 1 (of Affiliation): dept. name of organization\\
%line 2: name of organization, acronyms acceptable\\
%line 3: City, Country\\
%line 4: Email: name@xyz.com}
%}

\maketitle
\begin{abstract}
Technical debt refers to the phenomena of taking shortcuts to achieve short term gain at the cost of higher maintenance efforts in the future. Recently, approaches were developed to detect technical debt through code comments, referred to as Self-Admitted Technical debt (SATD). Due to its importance, several studies have focused on the detection of SATD and examined its impact on software quality. However, preliminary findings showed that in some cases SATD may live in a project for a long time, i.e., more than 10 years. These findings clearly show that not all SATD may be regarded as `bad' and some SATD needs to be removed, while other SATD may be fine to take on. Hence, the question becomes - which SATD should be removed?

Therefore, in this paper, we study the removal of SATD. In an empirical study on five open source projects, we examine how much SATD is removed and who removes SATD? We also investigate for how long SATD lives in a project and what activities lead to the removal of SATD? Our findings indicate that, on average, 74.9\% of SATD is removed, that aprox. 54.3\% is self-removed (i.e., removed by the same person that introduced it) and that self-removed SATD is addressed within 18 days, whereas non-self-removed SATD takes 127 days to address, on median.
\end{abstract}

\begin{IEEEkeywords}
Self-Admitted Technical Debt, Source Code Quality, Mining Software Repositories
\end{IEEEkeywords}


\section{Introduction}
\label{sec:introduction}
% -*- root: main.tex -*-
% !TEX root = main.tex

%overview technical debt
The term technical debt was first coined by Cunningham in 1993 to refer to the phenomena of taking a shortcut to achieve short term development gain at the cost of increased maintenance effort in the future \cite{Cunningham1992WPM}. The technical debt community has studied many aspects of technical debt, including its detection \cite{Zazworka2013EASE}, impact \cite{Zazworka2011MTD} and the appearance of technical debt in the form of code smells \cite{Fontana2012MTD}. 
Most recently, the notion of self-admitted technical debt (SATD) has been introduced by Potdar and Shihab~\cite{Potdar2014ICSME}: SATD refers to the situation when developers clearly indicate presence of
technical debt in the system implementation artifacts such as source code comments.
%researchers used source code comments to identify technical debt that referred to as
%\rabe{Everton} developed an approach to identify technical debt from code comments, referred to as self-admitted technical debt (SATD). 
SATD refers to the situation where developers know that the current implementation is not optimal and write comments alerting the inadequacy of the solution. 




%However, much technical debt remains in the projects. Why is studying removal important
Eventhough previous work argues that SATD has negative impact on software~\cite{Wehaibi2016SANER,kameiusingTDA2016}, it has also showed that some SATD remains in a project for long periods of time (up to 10 years\alexander{where does this data come from? isn't this an answer to RQ3?}) after its introduction. Therefore, an important question becomes ``why does some SATD remain for so long?'' and whether ``all SATD needs to be paid back?''. Examining the removal of SATD can shed light on potentially healthy patterns of debt, that may need not be paid back.

%What we do in this paper
Hence, in this paper we perform a mix-match empirical study of large open source software projects, and examine phenomena relating to the removal of SATD. In particular, we examine the following questions:
\begin{itemize}
	\item[\textbf{RQ}]\textbf{1:} \emph{How much self-admitted technical debt gets removed?} Non-removal of SATD suggests relative lack of importance of SATD for the developers. 
	\item[\textbf{RQ}]\textbf{2:} \emph{Who removes self-admitted technical debt? Is it most likely to be self-removed or removed by others?} One would expect the person that introduced SATD is better aware of the presence of SATD, and, hence, a priori, is more likely to remove SATD, i.e., to pay it back.
	\item[\textbf{RQ}]\textbf{3:} \emph{How long does self-admitted technical debt survive in a project?} Continuing the distinction between developers removing their own SATD as opposed to those removing SATD introduced by others, we would expect the former to remove SATD faster than the latter.
	\item[\textbf{RQ}]\textbf{4:} \emph{What activities lead to the removal of self-admitted technical debt?} Developers conduct both activities such as refactoring or code improvement that might explicitly target removal of technical debt, and activities related to new functionality or bug fixing that might lead to SATD removal as a byproduct.
\end{itemize}
To answer those questions, we first devise a technique to determine SATD introduction and removal. In total, we examine 5,733 SATD removals in five large open source systems.


\rabe{To answer these research questions, We use a twofold research method. We first  devise a technique to determine SATD introduction and removal. In total, we examine 5,733 SATD removals in five large open source systems. Then, we survey  15 \% open source developers developers to answers RQ4.}

%Our findings
\textbf{Paper contribution:} Our findings indicate that 
\begin{itemize}
	\item We find that the majority of \SATD comments gets removed over the evolution of the project. Considering all projects the percentage of \SATD comments removal ranges between 40.5--90.6\%, and on average 74.9\% of the identified \SATD is removed.
	\item We find that most \SATD (on average 54.4\% and median 61.0\%) is self-removed.
	\item We find that the median amount of time that self-removed technical debt stays in the project is 18 days whereas for the non-self-removal it is 127 days. In addition, we find that technical debt removals happens more when the files containing debt are edited than when they are removed.
	\item Developer add \SATD to track potential future bugs, code that needs improvements or areas to implement new features. Developers mostly remove \SATD when they are fixing bugs or adding new features. Very seldom do developers remove \SATD as part of refactoring efforts or dedicated code improvement activities.
\end{itemize}


\rabe{These finding highlight the self-admitted technical debt indicated in the source code comment are important development practice. So, developers go back and remove them.  }



%Organization
The rest of the paper is organized as follows: after detailing the case study setup in Section~\ref{sec:approach} we presents the results in Section~\ref{sec:case_study_results} and discuss their implications in Section~\ref{sec:discussion}. We position our results with respect to the related work in Section~\ref{sec:related_work} and evaluate threats to validity in Section~\ref{sec:threats_to_validity}. Finally, Section~\ref{sec:conclusion} concludes and sketches future work.







% \section{Motivating Example}
% \label{sec:motivating_example}
% \input{motivation}

\section{Case Study Setup}
\label{sec:approach}
% -*- root: main.tex -*-

The main goal of our study is to understand what happens with \SATD comments after they are introduced in projects. To do that, we divided our study in two main parts. First, we use a manually classified dataset that contains \SATD comments from three open source projects. Then, we trace each \SATD finding when and by who the technical debt was introduced and removed. After that, we run our analysis and examine the results. Second, to scale our approach to more projects we implement a process that does not depends on a manually classified dataset. Using this approach we extracted \SATD comments from other five open source projects, and similarly we analyze the \SATD comments of these projects. Figure~\ref{fig:manually_classified_data_approach_overview} shows an overview of our manual approach, Figure~\ref{fig:automatically_classified_data_approach_overview} shows an overview of our automatic approach, and the following subsections detail each step.

\begin{figure*}[thb!]
  \centering
  \includegraphics[width=1\textwidth]{figures/manually_classified_data_approach.pdf}
  \caption{Manually Classified Data Approach Overview}
  \label{fig:manually_classified_data_approach_overview}
\end{figure*}

\subsection*{Manually Classified Data Approach}
\label{sub:manually_classified_data_approach}

As shown in previous work, technical debt can be classified into different types ~\cite{Alves2014MTD}. However, design technical debt is the most common ~\cite{Maldonado2015MTD} and impactful ~\cite{Ernst2015FSE} type of debt. Therefore, to perform our study, we use manually classified \SATD comments from three different projects, namely Apache Ant, Apache Jmeter and Jruby. The analyzed dataset consists of 754 \SATD design comments distributed between the three projects. We choose to analyze Apache Ant, Apache Jmeter and Jruby as the version that contains the manually classified comments has enough past and future versions to be analyzed. Moreover, they have git repositories that are currently maintained enabling us to apply our approach. 

The manually classified comments are part of a bigger dataset of \SATD comments created during ours previous studies ~\cite{Maldonado2015MTD,Maldonado2015TSE}. Basically, during these previous works, we created a public available dataset containing 62,566 comments extracted from ten open source projects. These comments were classified as \SATD comments or as regular comments (i.e., comments without technical debt). The dataset was classified by the first author and later, to mitigate the risk of bias, another student was asked to classify a statistically significant sample of the dataset. The Cohen's kappa coefficient ~\cite{cohen1960coefficient} (i.e., the level of agreement between both reviewer) was of +0.81. The resulting coefficient is scaled to range between -1 and +1, where negative value means poorer than chance agreement, zero indicates exactly chance agreement, and positive value indicates better than chance agreement ~\cite{fleiss1973equivalence}.

\subsubsection*{Technical Debt Files Identification}
\label{subsub:technical_debt_files_identification}
The manually classified dataset contains the fully qualified name (i.e., file path plus the file name) for each one of the files that contains at least one of the 754 \SATD design debt comments. However, there is no guarantee that the file will not be moved in future versions. Therefore, we need to know the file path in the latest future version that the file was available. falar do checouk interativo fazendo o matching dos nomes

falar sobre todas versoes que temos que fazer checkout , do passado que fazemos isso identificando o nome dos arquivos e depois fazemos a mesma coisa pra versoes futuras . talvez podemos falar que o follow funciona pro passado mas nao pro futuro , ou como nossas versoes analizadas estao no meio da vida dos projetos nos escolhemos determinar os arquivos pelo nome no passado e no futuro. 

\subsubsection*{Checkout All Versions of Technical Debt Files}
\label{subsub:checkout_all_versions_of_technical_debt_files}

\subsubsection*{Identify Author Who Introduced the Technical Debt}
\label{subsub:identify_author_who_introduced_the_technical_debt}
more strait forward , first found

\subsubsection*{Identify Author Who Removed the Technical Debt}
\label{subsub:identify_author_who_removed_the_technical_debt}
need to add the strategy for removed files

\begin{figure*}[thb!]
  \centering
  \includegraphics[width=1\textwidth]{figures/automatically_classified_data_approach.pdf}
  \caption{Automatically Classified Data Approach Overview}
  \label{fig:automatically_classified_data_approach_overview}
\end{figure*}

\subsection*{Automatically Classified Data Approach}
\label{sub:automatically_classified_data_approach}

\subsubsection*{Project Data Extraction}
\label{subsub:project_data_extraction}

\subsubsection*{Checkout All Versions of Files}
\label{subsub:checkout_all_versions_of_files}

\subsubsection*{Parse Source Code}
\label{subsub:parse_source_code}

\subsubsection*{Filtering Comments}
\label{subsub:filtering_comments}

\subsubsection*{NLP Classification}
\label{subsub:nlp_classification}

\subsubsection*{Find Technical Debt Authors}
\label{subsub:find_technical_debt_authors}




\section{Case study Results}
\label{sec:case_study_results}

% -*- root: main.tex -*-
% !TEX root = main.tex
The main goal of our study is to better understand what happens to \SATD once it is introduced into software projects. To do so, our first step is to quantify how much of \SATD comments gets removed (RQ1). Next, we analyze who removes \SATD, i.e., if the same developer that introduced the debt is also most likely to remove it (RQ2). Then, we investigate how long the \SATD remains in the project (RQ3). Finally, we conduct a survey with 12 developers to understand why \SATD is introduced and removed (RQ4). 


%\alexander{An option here would also be to confirm the findings by interviewing/surveying the developers.}

%\alexander{shouldn't the following discussion of the RQs be in the introduction?}
%\emad{we incorporate a part of that in the intro.}
\subsection*{\rqi}
\subsubsection*{Motivation} Previous work showed that technical debt is widespread, unavoidable, and has a negative impact on the quality of software projects~\cite{Lim2012Software}. Therefore, a priori we expect that removing technical debt is a concern for developers. To understand how developers deal with technical debt we must first quantify how much debt is removed. 
%Also, answering this question will provide us insight if source code comments indeed helps developers to manage technical debt. 


\subsubsection*{Approach} To answer this question we automatically identify \SATD from the five analyzed projects. As described in Section~\ref{sub:checkout_all_versions_of_files}, we stored all versions of all source code files. Then, for each analyzed \SATD comment we take the oldest file version available in which the debt was found and incrementally search for matches in future versions of the file. The first time that the analyzed \SATD comment(s) appears in a file, indicates the exact file version that the \SATD comment was introduced. To analyze if the introduced \SATD comment was later removed, we search for the comment in the remaining file versions. When the comment is no longer found, we mark that version of the file as the removal version. In certain cases, a \SATD comment is found in one version only (i.e., the version that it is introduced in). Such cases indicate a scenario where the \SATD was introduced and removed immediately after. 

%Therefore, the first file version that we are not able to find a match for the analyzed comment is also the version that the \SATD comment was removed. Another possible way to remove a \SATD comment is by deleting the file that contains it. 


\subsubsection*{Results} Table \ref{tbl:removed_self_admitted_technical_debt_per_project} presents the identified and removed \SATD comments. We find that the majority (\textit{i.e.,} on average 74.4\%, median 76.7\%) of the identified \SATD comments were removed. We measure the average on a per project basis, \textit{i.e.,} the total from each project is taken and the average over the five projects in provided. For example, we were able to find 854 unique instances of \SATD comments when analyzing Ant project. 85.2\% (\textit{i.e.,} 738) of these \SATD comments were removed during the evolution of the project. Camel had the highest \SATD comments removal percentage (\textit{i.e.,} 90.6), whereas Hadoop had the lowest removal percentage achieving 40.5\%. 

Our findings indicate that developers tend to be aware and do care about \SATD. This finding corroborates with the survey findings of Ernst \emph{et al.}~\cite{Ernst2015FSE}.

\begin{table}[!t]
	\begin{center}
		\caption{Removed Self-Admitted Technical Debt per Project}
		\label{tbl:removed_self_admitted_technical_debt_per_project}
		\begin{tabular}{l|rrrr}
			\toprule
			\textbf{\thead{Project}} & \textbf{\thead{\#identified}} & \textbf{\thead{\#removed}} & \textbf{\thead{\% removed}} &  \textbf{\thead{\% remains}}  \\ 
			\midrule
			%\textbf{Ant   }  &  854    & 728    & 85.2 \\  
			\textbf{Camel }  &  4,331  & 3,926  & 90.6  & 9.4\\
			\textbf{Gerrit}  &  271    & 208    & 76.7 & 23.3 \\
			\textbf{Hadoop}  &  1,164  & 472    & 40.5 & 59.5 \\  
			%\textbf{Jmeter}  &  1,260  & 981    & 77.8 \\ 
			\textbf{Log4j }  &  135    & 118    & 87.4 & 12.6\\ 
			\textbf{Tomcat}  &  1,317  & 1,009  & 76.6 & 23.4\\   
			\midrule
			\textbf{Average} & -       & -      & 74.4 & 25.6\\
			\textbf{Median} & -       & -      & 76.7 & 23.3\\
			\bottomrule
		\end{tabular}
	\end{center}    
\end{table}

\begin{table}[t]
	\begin{center}
		\caption{Self-Removed Technical Debt per Project}
		\label{tbl:self_removed_technical_debt_vs_non_self_removed_technical_debt_per_project}
		\begin{tabular}{l|rrr}% | c c}
			\toprule
			\textbf{\thead{Project}} & \textbf{\thead{\#removed}} & \textbf{\thead{\#self-removed}} & \textbf{\thead{\% self-removed} }\\
			%& \textbf{\thead{\#non self-removed TD}} & \textbf{\thead{\% non self-removed TD}} \\ 
			\midrule
			%\textbf{Ant   }   & 728   &  372  & 51.09 &   356  & 48.90 \\  
			\textbf{Camel }   & 3,926 & 2,652 & 67.5 \\%&  1,274 & 32.5 \\  
			\textbf{Gerrit}   & 208   &  149  & 71.6 \\%&    59  & 28.4 \\  
			\textbf{Hadoop}   & 472   &  116  & 24.6 \\%&   356  & 75.4 \\  
			%\textbf{Jmeter}   & 981   &  663  & 67.58 &   318  & 32.41 \\  
			\textbf{Log4j }   & 118   &   72  & 61.0 \\%&    46  & 39.0 \\  
			\textbf{Tomcat}   & 1,009 &  578  & 57.3 \\%&   431  & 42.7 \\  
			\midrule
			\textbf{Average} & -      & -     & 54.4\\% &    -   & 43.6 \\
			\textbf{Median}  & -      & -     & 61.0 \\%&    -   & 39.0 \\
			\bottomrule
		\end{tabular}
	\end{center}    
\end{table}
\conclusion{The majority of \SATD comments are removed over time. In our five case study projects, between 40.5--90.6\% (median 76.7\%) of the identified \SATD is removed.}
 
%\alexander{Here we need a baseline comparison: are \SATD comments being removed more or less often than regular comments.}
%\emad{I don't think we need a baseline since we are not necessarly comparing to anything, we are just studying how much gets removed.}

\subsection*{\rqii}
%\vspace{3mm}

\subsubsection*{Motivation} As opposed to the technical debt in general, \emph{self-admitted} technical debt stands for technical debt ``confessed'' by the developers themselves. This intuition leads us to believe that it would be natural that the developers who expressed concern about the code would be also the ones who fix it in the future. However, it is unknown whether this is the case. It makes intuitive sense that self-removal of \SATD is easier, since the developers know about the reason for the \SATD introduction and possibly how to address it. The findings of this question have implications on the way that developers/manager/projects need to manage \SATD. For example, if it is found that \SATD is mostly addressed by others, then projects need to pay special attend to how this technical debt (and the areas of the code that it exists in) is documented. If on the other hand, it is indeed mostly self-removed, then the problem is less troubling.

%will  serves as an indication for developers that introduce \SATD to document it well since most of the time it maybe someone else who needs to address the \SATD.

%For example, it would be more difficult to have other developers to address \SATD since 1) they need to know how to address the issue and 2) they may not know where the \SATD is. 


\subsubsection*{Approach} To answer this question we analyzed the authors of the changes (\textit{i.e.,} commits from the source code repository) that introduced or removed \SATD comments. In order to do that, we first determine the commit in which a \SATD comment was added, then we check the further file versions to determine if there is any commit that removed the \SATD comment. Finally, we compare the authors of the commits to see if they are the same or not. 

We take into consideration two attributes of the change when comparing authors---the author name and email address. This is a necessary heuristic to mitigate the risk of misclassifying authors that change their names in the source code repository during the evolution of the project.  \alexander{didn't we have a complete list of aliases somewhere? or cannot we assume that this list is complete?}


\subsubsection*{Results} Table~\ref{tbl:self_removed_technical_debt_vs_non_self_removed_technical_debt_per_project} shows that in most cases, the majority of \SATD is removed by the same author who introduced it, referred to as ``self-removed technical debt''. On average, 54.4\% of all removals are self-removed and in four of the five projects, self-removal accounts for more than 50\%. Once again, we measure the average on a per project basis, \emph{i.e.}, the total from each project is taken and the average over the five projects in provided. The project with highest percentage of self-removed technical debt was Gerrit with 71.6\%, with the lowest percentage---Hadoop with 24.6\%. %Non self-removed technical debt, \emph{i.e.}, debt that was introduced by one author and removed by a different author in the future, represents on average 43.6\% = 100 - 54.4\% of the removals. 

Hadoop tends to be an outlier in terms of self-removed \SATD, however, it is worth mentioning that Hadoop had the least amount of removals overall (only 40.5\% of the \SATD is ever removed). There are many possible reasons for the low removal rates, \emph{e.g.}, high developer churn or lack of process to deal with technical debt. Although we shed some light on the potential reasons for the removal of \SATD later in RQ4, we believe that determining the exact reasons for \SATD removal warrant a study on its own.

% \alexander{Can we include this information in the previous table?}

%\alexander{The risk here is that the self-removal is not a consequence of the ``guilt confession'' but a side effect of one developer/small group of developers doing all the work in the project or in its part. Can we somehow check this?} 
%\emad{interesting. we should add this to the threats.}

\conclusion{Majority of \SATD is self-removed.}
%\alexander{I'm not a big fan of averages...}




\subsection*{\rqiii}
%\vspace{3mm}

\subsubsection*{Motivation} We know that the majority of \SATD is removed and most of the time it is removed by the same developer who introduced it. Next, we would like to know how long \SATD lives in a project before it is actually removed. Answering this question helps us to understand for how long it is normal to have \SATD comments in the projects. In addition, once we quantify the number of self-removed technical debt and the number of non self-removed technical debt comments, we would like to understand if these two categories of removal have differences between them. For example, since we know that the majority of \SATD is self-removed, is it the case that it is removed faster since our intuition is that it would be easier to address. 

\subsubsection*{Approach} To determine the amount of time that \SATD lives in a project, we use the time difference between the commit that introduces and removes the \SATD comment. The steps to identify the \SATD introducing and removing commits are the same as we outlined in RQs 1 and 2. We measure the average and median time for \SATD to be removed. Additionally, we generate survival plots for the removal of \SATD to determine how likely the technical debt will live in a project. Survival plots show the (general) trend times for a given even to occur. In our case, the survival plots show the  percentage of self-admitted technical debt that survives in a project overtime. Finally, we distinguish between self- and non-self-removed technical debt and compare the removal time of each. We compare the two distributions (\emph{i.e.}, self- and non-self-removal) using a Mann-Whitney test~\cite{mann1947test} to determine if the difference is statistically significant at the customary level of 0.05.

%find all changes that introduced a \SATD comment, and similarly, whenever possible, the change that removed the \SATD comment. Each change committed to the source code repository possesses a collection of useful information \alexander{please rephrase} that allows us to run a number of different analyses. To answer this question we examine the time between introduction and removal of each change related related to the \SATD comments that we previously were able to identify. In addition, we leverage the knowledge gathered so far to determine the authors of the change. 
%\alexander{We should perform a statistical analysis...}


%We also estimated the magnitude of the difference between self-removed technical debt and non self-removed technical debt using the Cliff's Delta (or $d$)~\cite{grissom2005effect}, a non-parametric effect size measure for ordinal data. We consider the effect size values: small for $d$ $<$ 0.33 (positive as well as negative values), medium for 0.33  $\leq d<$ 0.474 and large for $d \geq$ 0.474.

\subsubsection*{Results} Figure~\ref{fig:removed_all_std_comments} shows the mean and median times for \SATD to be removed from the respective projects. The distribution of \SATD removal is skewed, as indicated by plots and the difference in the mean and median removal times. In general, the time that \SATD stays in a project  varies from one project to another and varies between 18.2--172.8 days on median and 82--613.2 days on average. One clear finding however, is that in Camel and Gerrit, \SATD is removed faster than in Hadoop, Log4j and Tomcat.

Figure~\ref{fig:survival_plots} shows the survival plots of \SATD for the five studied projects. 
Survival plot is a technique originating from the medical domain indicating the probability of 
a patient to survive at least for $x$ days. To estimate this probability one would ideally like to 
have complete information about the death time of all patients. Such an assumption is, however,
usually not realistic as some patients might still be alive at the end of the observations, \emph{i.e.},
the data is right-censored. Kaplan
and Meier~\cite{kaplan1958nonparametric} have proposed a technique to estimate the survival
in presence of right-censored data. As we have seen in Table~\ref{tbl:removed_self_admitted_technical_debt_per_project} some \SATD comments remained at
the end of the observations, \emph{i.e.}, our data is also right-censored, in Figure~\ref{fig:survival_plots}
we present the Kaplan-Meier estimators. The use of Kaplan-Meier estimators is common in 
software evolution applications of survival analysis~\cite{samoladas2010survival,goeminne2015towards}.

Inspecting Figure~\ref{fig:survival_plots} we observe that for all projects, there is a steep decline in the first few hundred days, suggesting that in all projects an important share of \SATD is rapidly removed. Projects do differ in how steep the drop is and where it flattens out. For example, for the Camel project, there is a steep drop in \SATD after around 126 days and a long tail after that. This means that in Camel, the likelihood of \SATD surviving (\emph{i.e.}, existing in the system after introduction) drops significantly after 126 days, and after that time, the chance of surviving is less than 20\%, as indicated by the survival function. Another extreme case is Hadoop, where the chance of \SATD surviving for more than 2,000 days is very high, close 59.5\%, the percentage
of the \SATD comments remaining at the end of the observations reported in Table~\ref{tbl:removed_self_admitted_technical_debt_per_project}. 

We also compare the time that self-removed and non self-removed \SATD exists in the system before it gets removed. We find that self-removed technical debt gets removed faster than non self-removed technical debt. Figure~\ref{fig:removal_self_vs_nonself} shows that, on median for all projects, self-removed technical debt is removed earlier than non-self-removed technical debt. Our finding confirms our intuition, however, the exact reasons (e.g., is it because the remover is more familiar with the debt) as to why self-removals take less time warrant a study on its own.\alexander{There is no Mann-Whitney here: either we report the results or we remove the discussion of M-W above. If we are doing statistics then we also need to report the effect size.}


\conclusion{The amount of time \SATD remains in a project before removal varies from one project to another and ranges between 18.2--172.8 days on median and 82--613.2 days on average. Moreover, self-removed technical debt is removed faster than non-self-removed technical debt.}
%presents the comparison between self-removal and non self-removal of three projects, namely, Gerrit, Hadoop and Tomcat. We selected these projects since they represent the general behaviour of the analyzed dataset. As we can observe in Figures \ref{fig:removal_comparison_gerrit}, \ref{fig:removal_comparison_hadoop} and \ref{fig:removal_comparison_tomcat} self-removal technical debt is removed faster than non self-removal technical debt when considering the amount of days that each technical debt was part of the analyzed system. The horizontal black line on each figure mark the median value of removal time. In addition for each figure we plotted the number of occurrences measured on each dataset, as shown in the legend indicated by the letter `N' followed by the precise median in days.

%\begin{table}[tbh!]
%	\centering
%	\caption{Mean and Median Time to Remove Saelf-Admitted Technical Debt per Project}
%	\label{tab:Distribution_of_the_removed_STD_Comments}
%	\begin{tabular}{@{}l|rrrr@{}}
%		\toprule
%		\textbf{Project} &~&\textbf{Median} &~& \textbf{Mean} \\ \midrule
%		\textbf{Camel} &~& 18.16 &~& 82.03 \\
%		\textbf{Gerrit} &~& 10.84 &~& 177.00 \\
%		\textbf{Hadoop} &~& 159.00 &~& 326.80 \\
%		\textbf{Log4j} &~& 172.80 &~& 516.00 \\
%		\textbf{Tomcat} &~& 164.90 &~& 613.20 \\			
%		%\midrule
%		
%		%\textbf{Mean} &~& 105.10 &~&343.00 \\
%		
%		%\textbf{Median} &~& 159.00 &~& 326.80 \\ 
%		\bottomrule
%	\end{tabular}
%\end{table}



%-------------------------------------------------------------------------------

\begin{figure}[t]
	\centering
	\includegraphics[width=0.9\columnwidth]{figures/test/removed_all_STD_comments.pdf}
	\caption{The distribution of all the removed STD comments}
	\label{fig:removed_all_std_comments}
\end{figure}


\begin{figure*}[t]
	\centering
	
	\begin{subfigure}[b]{0.31\textwidth}
		\includegraphics[width=\textwidth]{figures/Survival/camel.pdf}
		%\caption{Self-Removal vs \\ Non Self-Removal for Hadoop}
		%\label{fig:removal_comparison_camel_survival}
	\end{subfigure}
	~
	~
	\begin{subfigure}[b]{0.31\textwidth}
		\includegraphics[width=\textwidth]{figures/Survival/gerrit.pdf}
		%\caption{Self-Removal vs \\ Non Self-Removal for Tomcat}
		%\label{fig:removal_comparison_tomcat_survival} 
	\end{subfigure}
	~
	~
	\begin{subfigure}[b]{0.31\textwidth}
		\includegraphics[width=\textwidth]{figures/Survival/hadoop.pdf}
		%\caption{Self-Removal vs \\ Non Self-Removal for Tomcat}
		%\label{fig:removal_comparison_gerrit_survival} 
	\end{subfigure}

	
	\begin{subfigure}[b]{0.31\textwidth}
		\includegraphics[width=\textwidth]{figures/Survival/log4j.pdf}
		%\caption{Self-Removal vs \\ Non Self-Removal for Hadoop}
		%\label{fig:removal_comparison_log4j_survival}
	\end{subfigure}
	~
	~
	\begin{subfigure}[b]{0.31\textwidth}
		\includegraphics[width=\textwidth]{figures/Survival/tomcat.pdf}
		%\caption{Self-Removal vs Non Self-Removal for Tomcat}
		%\label{fig:removal_comparison_hadoop_survival} 
	\end{subfigure}
		\caption{Survival plots show the probability of the removal of STD Comment for all studied projects}
		\label{fig:survival_plots}
\end{figure*}

%------------------------------------------------------------------------------

\begin{figure*}[t]
	\centering
	%\begin{subfigure}[b]{0.24\textwidth}
	%	\includegraphics[width=\textwidth]{figures/test/Ant.pdf}
		%\caption{Self-Removal vs \\ Non Self-Removal for Gerrit}
	%	\label{fig:removal_comparison_gerrit}
	%\end{subfigure}
	%\begin{subfigure}[b]{0.24\textwidth}
	%	\includegraphics[width=\textwidth]{figures/test/Jmeter.pdf}
	%\caption{Self-Removal vs \\ Non Self-Removal for Gerrit}
	%	\label{fig:removal_comparison_gerrit}
	%\end{subfigure}
	\begin{subfigure}[b]{0.195\textwidth}
		\includegraphics[width=\textwidth]{figures/test/Camel.pdf}
		%\caption{Self-Removal vs \\ Non Self-Removal for Hadoop}
		\label{fig:removal_comparison_camel}
	\end{subfigure}
	\begin{subfigure}[b]{0.193\textwidth}
		\includegraphics[width=\textwidth]{figures/test/Gerrit.pdf}
		%\caption{Self-Removal vs \\ Non Self-Removal for Tomcat}
		\label{fig:removal_comparison_gerrit} 
	\end{subfigure}
	\begin{subfigure}[b]{0.195\textwidth}
		\includegraphics[width=\textwidth]{figures/test/Hadoop.pdf}
		%\caption{Self-Removal vs Non Self-Removal for Tomcat}
		\label{fig:removal_comparison_hadoop} 
	\end{subfigure}
	\begin{subfigure}[b]{0.191\textwidth}
		\includegraphics[width=\textwidth]{figures/test/Log4j.pdf}
		%\caption{Self-Removal vs \\ Non Self-Removal for Hadoop}
		\label{fig:removal_comparison_log4j}
	\end{subfigure}
	\begin{subfigure}[b]{0.195\textwidth}
		\includegraphics[width=\textwidth]{figures/test/Tomcat.pdf}
		%\caption{Self-Removal vs \\ Non Self-Removal for Tomcat}
		\label{fig:removal_comparison_tomcat} 
	\end{subfigure}
	\begin{subfigure}[b]{0.30\textwidth}
		\includegraphics[width=\textwidth]{figures/test/legend_.pdf}
		%\caption{Self-Removal vs \\ Non Self-Removal for Tomcat}
		%\label{fig:removal_comparison_tomcat} 
	\end{subfigure}
	\caption{Self-Removal vs Non Self-Removal for all studied projects}
	\label{fig:removal_self_vs_nonself}
\end{figure*}





%--------------------------------------------------------------------------------



%\begin{table}[t]
%	\begin{center}
%		\caption{Self-Removal vs Non Self-Removal: Mann-Whitney Test ($p$-value) and Cliff's Delta ($d$) \emad{is this actually discussed?}}
%		\label{tbl:statistic}
%		\begin{tabular}{l| rrr}
%			\toprule
%			\textbf{\thead{Project}} & \textbf{\thead{$p$-value}}&~~~ & \textbf{\thead{$d$}}\\ 
%			\midrule
%			%\textbf{Ant   }   &  < 2.2e-16& ~~~  &  -0.427(medium)  \\  
%			\textbf{Camel }   &  0.0001253& ~~~ &  -0.075(negligible)  \\  
%			\textbf{Gerrit}   &  3.581e-14& ~~~ &  -0.671(large)  \\  
%			\textbf{Hadoop}   &  < 2.2e-16& ~~~ &  -0.531(large)  \\  
%			%\textbf{Jmeter}   &  < 2.2e-16& ~~~ &  -0.325(small)  \\  
%			\textbf{Log4j}   &  2.345e-06 & ~~~ &  -0.517(large)  \\  
%			\textbf{Tomcat}   &  2.2e-16  & ~~~ &  -0.820(large) \\  
%			\bottomrule
%		\end{tabular}
%	\end{center}    
%\end{table}





%One other thing that we evaluate to answer this question is the general nature of the removals. We divided all the removals into two categories. The first we call file edition and the second file removal. In other words, we want to better understand if the removals of technical debt are happening as the project evolves and resembles a ''floss refactoring'' type of maintenance or if the files just gets deleted after a time due to its entropy. We find that the majority of the removals happens during the edition of files and not when the files get deleted. On average 74.8\% of all removals are done by inclemently editing the files that contains technical debt. \emad{Why do we do this analysis?}



% \begin{itemize}
% \item Add figures of distribution of removal time.
% \item Add survival plots
% \item file addition vs. removal
% \item Difference between self vs. non-self removal.
% \end{itemize}



\begin{figure}[!tb]
	\centering
	\includegraphics[width=\columnwidth]{figures/test/responses_question.pdf}
	\caption{Survey responses on how often do developers encounter, add and address self-admitted technical debt.}
	\label{fig:encouner_add_address}
\end{figure}




 \noindent\rqiv
 %\vspace{3mm}

\subsubsection*{Motivation} 
Thus far, our analysis has been quantitative in nature. To triangulate our findings and better understand our findings, we would like to perform complementary qualitative analysis to understand the experiences and motives of developers who introduce and remove \SATD.
%So far, we found that the majority of TD comments gets removed over the evolution of the projects, and that they could be removed by the same developers who wrote them or other developers. These fining raises two main questions what type of development activities lead to the introduce and remove of self-admitted technical debt comments. Answering these questions, we could understand why developers add such technical debt comment into the source code, and if there are special development activities that developers performer to address these comments. 


 
\subsubsection*{Approach} 
To understand the activities that lead to the introduction and removal of \SATD, we designed and sent an online survey. The survey included three main sections: 1)  questions regarding the participant's role and development tasks and experience in the project, 2) three multiple choice, likert-scale, questions about how often developers encounter, add, and address self-admitted technical debt, and 3) two open ended questions asking why developers add or remove \SATD. 

To identify the population of our survey, we collected the names and emails of all developers who added or removed self-admitted technical debt in our dataset. In total, we found 250 unique developers from the studied projects and we successfully sent the survey to 188 of them.  We received 14 responses to our survey. Although this is smaller than usual response rate, the area of technical debt is difficult to discuss, especially since some developers may feel they will be negatively perceived. For, the open-ended questions, we manually analyzed the free-text answers and identified 6 main reasons why developers add self-admitted technical debt and 5 main reasons for removing self-admitted technical debt.

Table~\ref{survey_responses} shows the participant's role in the projects, their main development tasks and developer experiences. Of the 14 participants, 8 (57\%) of them identified themselves as core developer, and 6 (43\%) are contributors to the projects. Five of the 14 participants work on fixing bug, and five work on implementing new features. Only one participant has the task of code reviewer. Another 3 participants (21\%) indicted having different tasks (e.g., project user). Approximately 86\% of the participants have more than five years of development experiences, and two participants have less than or equal to five yeas of development experiences.

%To address this research question, we analyze the developer responses to the online survey described in section~\ref{Survey_Design}. The survey has two main sections: the first part asks questions related to likert-scale type questions about how often developers encounter, add and address technical debt comments; the second part contains two open-ended questions about the activities that developers perform to add and address self-admitted technical debt. 


\begin{table}[t]
	\centering
	\caption{Background of participants in online survey}
	\label{survey_responses}
	\begin{tabular}{@{}l|p{1.2in}|l|l|l@{}}
		\toprule
		\multirow{2}{*}{\textbf{\begin{tabular}[c]{@{}l@{}}Developer\\ Role\end{tabular}}} & \multirow{2}{*}{\textbf{\begin{tabular}[c]{@{}l@{}}Developer\\ Task\end{tabular}}} & \multicolumn{3}{c}{\textbf{\begin{tabular}[c]{@{}c@{}}Developer Exp. \\ (in years)\end{tabular}}} \\ \cmidrule(l){3-5} 
		&  & \textbf{1-2} & \textbf{3-5} & \textbf{\textgreater5} \\ \midrule
		\begin{tabular}[c]{@{}c@{}}A Contributor\\ Developer (8)\end{tabular} & BF, BF, NF, OTH, BF, BF, OTH, NF & 1 & 0 & 7 \\
		\begin{tabular}[c]{@{}l@{}}A Core\\ Developer (6)\end{tabular} & NF, BF, OTH, NF, CR, NF & 0 & 1 & 5 \\ \bottomrule
			
			
	\end{tabular}
	
	BF=Bug Fixing, NF=New Feature, CR=Code Review, OTH=Other.
\end{table}





\subsubsection*{Results} 
Figure~\ref{fig:encouner_add_address} shows the results of the likert-scale questions about how often developers encounter, add, address self-admitted technical debt. Developers mostly agreed that they encounter source code comments indicting self-admitted technical debt. In 50\% of the responses, developers stated that they sometimes (i.e., once a month) see such comments, and the other 50\% of them said they encounter such comments often or very often (i.e., everyday). When adding self-admitted technical debt, developer's responses vary from often (i.e., once a week) adding self-admitted technical debt (approx. 21\%) to rarely (approx. 36\%). Another 43\% said that they sometimes added such comments. We also asked developers how often they address self-admitted technical debt. 50\% of the developers mentioned that they rarely address self-admitted technical debt, 29\% of respondents mentioned that they sometimes address self-admitted technical debt, and 21\% of the respondents mentioned that they often address self-admitted technical debt. The finding is inline with our quantitative findings that approximately 50\% of \SATD is removed.

As for why developers tend to add \SATD, approximately 64\% of developers (P1, P4, P5, P8, P9, P11, P12, P13, and P14) indicted that they add self-admitted technical debt as \emph{a tracker in the source code for potential bugs or source code that needs to be improved or document a need for a new feature}. For example, P12 states that \textit{"It is usually a marker in the source of a missing feature or known bug."}. Also, contrary to Postdar and Shihab~\cite{Potdar2014ICSME} we find that developers add self-admitted technical debt because of time pressure (approx. 36\% - P1, P2, P7, P13, and P14) to deliver tasks. For example, P1 \textit{"Because they want to deliver, and when balancing an early delivery against technical debt."} Some other reasons for adding self-admitted technical debt are very rare and are only mentioned once or twice (e.g,. a remainder or looking for feedback). For example, P5 said that \textit{"They are not sure about the effects of their code and want feedback..."}.

In response to the question on why developers address self-admitted technical debt, we identified five reasons. The most cited reason for addressing self-admitted technical debt is \emph{to fix bugs} (around 64\% - P1, P4, P5, P7, P8, P9, P10, P12 and P13). For example, P12 states \textit{", usually as part of fixing a user-reported issue..."} The second most frequent reasons is to add a new feature (36\% - P1, P4, P6, P12, and P14) and improve the code overall (36\% - P7, P8, P9, P10, and P11). The other two, less frequent, reasons are addressing \SATD when refactoring code (14\% - P7 and P9) and to provide a generally better solution (14\% - P2 and P7). Our findings indicate that there is a need for software projects to allocate resources to specifically address \SATD, since most respondents do not seem to indicate that a systematic process is in place to address \SATD. And, in most cases it seems like dealing with \SATD is done in an ad-hoc manner.



 \conclusion{Developer add \SATD to track potential future bugs, code that needs improvements or areas to implement new features. Developers mostly remove \SATD when they are fixing bugs or adding new features. Very seldom do developers remove \SATD as part of refactoring efforts or dedicated code improvement activities.}
 
 
 
% % Please add the following required packages to your document preamble:
% % \usepackage{booktabs}
% \begin{table*}[ht!]
% 	\centering
% 	\caption{My caption}
% 	\label{my-label}
% 	\begin{tabular}{@{}lrrrrr@{}}
% 		\toprule
% 		\textbf{} & \textbf{Never} & \textbf{Rarely} & \textbf{Sometimes} & \textbf{Often} & \textbf{Very Often} \\ \midrule
% 		How often do developers encounter technical debt comments? & 0.0 & 0.0 & 50.0 & 21.4 & 28.6 \\
% 		How often do developers add technical debt comments? & 7.1 & 28.6 & 42.9 & 7.1 & 14.3 \\
% 		How often do developers address technical debt comments? & 0.0 & 50.0 & 28.6 & 21.4 & 0.0 \\ \bottomrule
% 	\end{tabular}
% \end{table*}
% 
 
 

\section{Discussion}
\label{sec:discussion}
% -*- root: main.tex -*-
% !TEX root = main.tex
%The notion of \SATD is based on the idea of admission, i.e., acknowledgement that a fact or statement is true\footnote{\url{http://www.merriam-webster.com/dictionary/admission}}. In a way, this can be seen as a postponed promise to fix a bug or to implement functionality.

Our study shows that more of the self-removed technical debt is \emph{usually} removed than of the non-self-removed technical debt,
and that this usually happens faster for the self-removed technical debt.
These findings suggest that developers feel a certain responsibility not only for indicating technical debt but also for removing it.\alexander{Interviews?}

\alexander{discuss practical implications of our findings}
Our findings have several implications for design of software development environments. 
In order to support self-removal of technical debt, those tools 

Jian and Hassan have studied removal of comments in PostgreSQL from 1996 to 2005~\cite{Jiang:Hassan}. They have observed that at each 30-day period 0--40\% of functions with header comments have been removed; similar observation has been made for functions with non-header comments. \alexander{It seems that this is the only reference on comment removal and this is not comparable with our results. Shall we keep it or drop it? Move it elsewhere?}




\section{Related Work}
\label{sec:related_work}
% -*- root: main.tex -*-

In this section, we describe the related work. We divided the related work into two sections: work related to the management and the detection of technical debt in general, and work related to the identification of self-admitted technical debt.

\subsection{The detection \& management of technical debt in general.} A number of earlier work studied the management and detection of technical debt in general. Seaman \textit{et al.}~\cite{Seaman2011}, Kruchten \textit{et al.}~\cite{Kruchten2013IWMTD} and Brown \textit{et al.}~\cite{Brown2010MTD} made several reflections about the term `technical debt' and mentioned that it is commonly used to communicate development issues to managers. Other work by Zazworka \textit{et al.}~\cite{Zazworka2013EASE} focused on the detection of technical debt, where they conducted experiments to compare the efficiency of automated tools in comparison with human elicitation in detecting technical debt. They found that there is a small overlap between the two approaches. They also concluded that automated tools are more efficient in finding defect debt, whereas developers can realize more abstract categories of technical debt. In a follow up work, Zazworka \textit{et al.}~\cite{Zazworka2011MTD} conducted a study to measure the impact of technical debt on software quality. They focused on a particular kind of debt, namely design debt measured using God classes. They found that God classes are more likely to change, and therefore, have a higher impact on software quality. Other work by Fontana \textit{et al.}~\cite{Fontana2012MTD} investigated design technical debt appearing in the form of code smells, namely: God Classes, Data Classes, and Duplicated Code. They proposed an approach to classify which one of the different code smells should be addressed first, based on their potential risk. Ernst \textit{et al.}~\cite{Ernst2015FSE} conducted a survey involving more than 1,800 participants and found that architectural decisions are the most important source of technical debt.



In an earlier study Klinger \textit{et al.} have conducted four interviews and observed that ``the individuals choosing to incur technical debt are usually different from those responsible for servicing the debt''~\cite{Klinger:etal}. 
This observation has been questioned by Spinola~\textit{et al.}, who have found that while the online-survey respondents tended to agree with the observation, the paper-survey respondents achieved high consensus in neither agreeing nor disagreeing with the observation~\cite{Spinola:etal}. Therefore, to complement these studies we analyze the source code.
Indeed, if the observation of Klinger~\textit{et al.} holds for \SATD then we expect to see a clear separation between the individuals introducing \SATD and individuals removing \SATD, resulting in a low \SATD self-removal.

Jian and Hassan have studied removal of comments in PostgreSQL from 1996 to 2005~\cite{Jiang:Hassan}. They have observed that at each 30-day period 0--40\% of functions with header comments have been removed; a similar observation has been made for functions with non-header comments. Unfortunately, different focus of our studies (comments vs. functions, \SATD-comments vs. any comments) render our results incomparable.


Our work differs from the work that uses code smells to detect design technical debt, since we use code comments to detect technical debt. Moreover, our study focuses on the removal of self-admitted technical debt, rather than its identification or management.




\subsection{The detection \& management of ``self-admitted'' technical Debt.} 
The work that is most related to ours is the work by Potdar and Shihab~\cite{Potdar2014ICSME} and Bavota and Russo~\cite{Bavota2016MSR}. Potdar and Shihab~\cite{Potdar2014ICSME} introduce the self-admitted technical debt. They extracted the code comments of four projects and analyzed more than 100K comments to come up with 62  patterns that indicate self-admitted technical debt. Their findings show that 2.4--31\% of the files in a project contain self-admitted technical debt. More specifically, they found that 1) the majority of the self-admitted technical debt is removed in the immediate next release; 2) developers with higher experience are mostly the ones who introduce the self-admitted technical debt; 3) release pressure does not play a major role in the introduction of self-admitted technical debt. 

Bavota and Russo~\cite{Bavota2016MSR} replicated the study of \SATD on a large set of Apache and Eclipse projects and confirmed the findings observed by Potdar and Shihab in their earlier work~\cite{Potdar2014ICSME}. Furthermore, they found that: 1) approximately 57\% of \SATD get removed during the change history of software projects, and 2) around 63\% of \SATD is removed by the same developers who introduced them, \emph{i.e.,} are self-removed. Our work differs from that by Potdar and Shihab~\cite{Potdar2014ICSME} and Bavota and Russo~\cite{Bavota2016MSR} in that we focus exclusively on the removal of \SATD. More specifically,  we use a more accurate technique to identify \SATD~\cite{Maldonado2015TSE}. In addition to quantify removal and examine who removes \SATD, we also examine how long \SATD tends to live in a project and shed light on the activities that lead to the removal of \SATD. In many ways, our study complements prior work on \SATD.

In recent work, Maldonado~\textit{et al.}~\cite{Maldonado2015TSE} used Natural Language Processing technical to identify self-admitted technical debt from source code comment. Their experiment showed that the proposed method achieves 90\% classification accuracy in identifying design and requirement self-admitted technical debt.  Maldonado and Shihab~\cite{Maldonado2015MTD} examined more than 33K comments to classify the different types of self-admitted technical debt found in source code comments. Other work~ Farias \textit{et al.}~\cite{Farias2015MTD} proposed a contextualized vocabulary model for identifying technical debt in comments using word classes and code tags in the process.



Other work studied the management and the impact of self-admitted technical debt. Wehaibi \textit{et al.}~\cite{Wehaibi2016SANER} examined the impact of self-admitted technical debt and found that self-admitted technical debt leads to more complex changes in the future. All three of the aforementioned studies used the comment patterns approach to detect self-admitted technical debt. Kamei~\textit{et al.}~\cite{kameiusingTDA2016}~proposed a method to measure technical debt interest using self-admitted technical debt comments in the source code. They found around 42\% of the technical debt in the studied projects incurs positive interest.


Similar to previous work, our work also uses code comments to detect self-admitted technical debt. However, we use an NLP technique to identify self-admitted technical debt in order to conduct our empirical study on the \emph{removal} of \SATD.














\section{Threats to Validity}
\label{sec:threats_to_validity}
\noindent~\textbf{\textit{Internal Validity}}

\noindent~\textbf{\textit{Construct Validity}}

manually classification of the commits. 

we consider commits as a single unit of change.

Identify the unique developers 


\noindent~\textbf{\textit{External Validity}}

\section{Conclusion and Future work}
\label{sec:conclusion}
% -*- root: main.tex -*-
% !TEX root = main.tex
Self-admitted technical debt refers to technical debt that can be detected through code comments. Prior work examined the detection, management and impact of \SATD. However, little is known about the removal of such technical debt. In this paper, we conduct an empirical study to examine how much \SATD is removed, how long such technical debt lives in a project before removal and who removes such debt. We find that the majority of \SATD is removed (74.4\% on average), that \SATD is mostly self-removed (54.4\% on average), and that it lasts between 82 - 613.2 days on average in a project before it is removed. Then, we conduct a survey with 14 developers to understand the reasons for the introduction and removal of \SATD. We find that there is no formal process to remove \SATD, and most removals occur as part of bug fixing.

Our results provide insights that indicate that \SATD is important, which is why the majority of it is removed. Also, they suggest that although developers are aware of the need to remove \SATD, most projects do not employ any formal process to address it. Hence, techniques are needed to allow projects to effectively and systematically address \SATD.

In the future, we plan to perform qualitative studies that examine the `whys' of our findings. In particular, we would like to examine why developers tend to self-remove technical debt. Additionally, we plan to better understand why some projects remove less \SATD than others. Finally, we plan to study the introduction and removal of \SATD of the projects at the revision level since that may provide a deeper understanding of the removal of \SATD.



\bibliographystyle{abbrv}
\balance
\bibliography{bibliography}  
\end{document}
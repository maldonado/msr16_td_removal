% -*- root: main.tex -*-
% !TEX root = main.tex

%overview technical debt
The term technical debt was first coined by Cunningham in 1993 to refer to the phenomena of taking a shortcut to achieve short term development gain at the cost of increased maintenance effort in the future \cite{Cunningham1992WPM}. The technical debt community has studied many aspects of technical debt, including its detection \cite{Zazworka2013EASE}, impact \cite{Zazworka2011MTD} and the appearance of technical debt in the form of code smells \cite{Fontana2012MTD}. 
Most recently, the notion of self-admitted technical debt (SATD) has been introduced by Potdar and Shihab~\cite{Potdar2014ICSME}: SATD refers to the situation when developers clearly indicate presence of
technical debt in the system implementation artifacts such as source code comments.
%researchers used source code comments to identify technical debt that referred to as
%\rabe{Everton} developed an approach to identify technical debt from code comments, referred to as self-admitted technical debt (SATD). 
SATD refers to the situation where developers know that the current implementation is not optimal and write comments alerting the inadequacy of the solution. 




%However, much technical debt remains in the projects. Why is studying removal important
Eventhough previous work argues that SATD has negative impact on software~\cite{Wehaibi2016SANER,kameiusingTDA2016}, it has also showed that some SATD remains in a project for long periods of time (up to 10 years\alexander{where does this data come from? isn't this an answer to RQ3?}) after its introduction. Therefore, an important question becomes ``why does some SATD remain for so long?'' and whether ``all SATD needs to be paid back?''. Examining the removal of SATD can shed light on potentially healthy patterns of debt, that may need not be paid back.

%What we do in this paper
Hence, in this paper we perform a mix-match empirical study of large open source software projects, and examine phenomena relating to the removal of SATD. In particular, we examine the following questions:
\begin{itemize}
	\item[\textbf{RQ}]\textbf{1:} \emph{How much self-admitted technical debt gets removed?} Non-removal of SATD suggests relative lack of importance of SATD for the developers. 
	\item[\textbf{RQ}]\textbf{2:} \emph{Who removes self-admitted technical debt? Is it most likely to be self-removed or removed by others?} One would expect the person that introduced SATD is better aware of the presence of SATD, and, hence, a priori, is more likely to remove SATD, i.e., to pay it back.
	\item[\textbf{RQ}]\textbf{3:} \emph{How long does self-admitted technical debt survive in a project?} Continuing the distinction between developers removing their own SATD as opposed to those removing SATD introduced by others, we would expect the former to remove SATD faster than the latter.
	\item[\textbf{RQ}]\textbf{4:} \emph{What activities lead to the removal of self-admitted technical debt?} Developers conduct both activities such as refactoring or code improvement that might explicitly target removal of technical debt, and activities related to new functionality or bug fixing that might lead to SATD removal as a byproduct.
\end{itemize}
To answer those questions, we first devise a technique to determine SATD introduction and removal. In total, we examine 5,733 SATD removals in five large open source systems.


\rabe{To answer these research questions, We use a twofold research method. We first  devise a technique to determine SATD introduction and removal. In total, we examine 5,733 SATD removals in five large open source systems. Then, we survey  15 \% open source developers developers to answers RQ4.}

%Our findings
\textbf{Paper contribution:} Our findings indicate that 
\begin{itemize}
	\item We find that the majority of \SATD comments gets removed over the evolution of the project. Considering all projects the percentage of \SATD comments removal ranges between 40.5--90.6\%, and on average 74.9\% of the identified \SATD is removed.
	\item We find that most \SATD (on average 54.4\% and median 61.0\%) is self-removed.
	\item We find that the median amount of time that self-removed technical debt stays in the project is 18 days whereas for the non-self-removal it is 127 days. In addition, we find that technical debt removals happens more when the files containing debt are edited than when they are removed.
	\item Developer add \SATD to track potential future bugs, code that needs improvements or areas to implement new features. Developers mostly remove \SATD when they are fixing bugs or adding new features. Very seldom do developers remove \SATD as part of refactoring efforts or dedicated code improvement activities.
\end{itemize}


\rabe{These finding highlight the self-admitted technical debt indicated in the source code comment are important development practice. So, developers go back and remove them.  }



%Organization
The rest of the paper is organized as follows: after detailing the case study setup in Section~\ref{sec:approach} we presents the results in Section~\ref{sec:case_study_results} and discuss their implications in Section~\ref{sec:discussion}. We position our results with respect to the related work in Section~\ref{sec:related_work} and evaluate threats to validity in Section~\ref{sec:threats_to_validity}. Finally, Section~\ref{sec:conclusion} concludes and sketches future work.






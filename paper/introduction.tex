% -*- root: main.tex -*-
% !TEX root = main.tex

%overview technical debt
The term technical debt was first coined by Cunningham in 1993 to refer to the phenomena of taking a shortcut to achieve short term development gain at the cost of increased maintenance effort in the future \cite{Cunningham1992WPM}. The technical debt community has studied many aspects of technical debt, including its detection \cite{Zazworka2013EASE}, impact \cite{Zazworka2011MTD} and the appearance of technical debt in the form of code smells \cite{Fontana2012MTD}. 
Most recently, the notion of self-admitted technical debt (SATD) has been introduced by Potdar and Shihab~\cite{Potdar2014ICSME}: SATD refers to the situation when developers clearly indicate presence of
technical debt in the system implementation artifacts such as source code comments.
%researchers used source code comments to identify technical debt that referred to as
%\rabe{Everton} developed an approach to identify technical debt from code comments, referred to as self-admitted technical debt (SATD). 
SATD refers to the situation where developers know that the current implementation is not optimal and write comments alerting the inadequacy of the solution. 


% What people did and what is the impact of TD. What they found.
In the last few years, an increasing amount of work has focused on SATD. In particular, prior work focused on the detection of SATD~\cite{Potdar2014ICSME} and the classification of different types of SATD and the development of datasets to enable future studies on SATD~\cite{Maldonado2015MTD}. Other work by Bavota and Russo~\cite{Bavota2016MSR} performed an empirical study of SATD on a large number of Apache projects showed that SATD is prevalent in open source projects, is long lived and is increasing over time. A study by Wehaibi et al.~\cite{Wehaibi2016SANER} examined the impact of SATD on quality and found that SATD does not necessarily relate to more defects, however, it does make the software system more complex.

%However, much technical debt remains in the projects. Why is studying removal important
Eventhough a plethora of work argues that SATD has negative impacts, prior work also showed that some SATD remains in a project for long periods of time (up to 10 years) after its introduction. Therefore, an important question becomes ``why does some SATD remain for so long?'' and whether ``all SATD needs to be paid back?''. Examining the removal of SATD can shed light on potentially healthy patterns of debt, that may need not be paid back.

%What we do in this paper
Hence, in this paper we perform an mix-match empirical study using seven large open source software projects, where we examine phenomena relating to the removal of SATD. In particular, we examine (RQ1) how much SATD is actually removed in these projects?, (RQ2) How long SATD survives in a software project? and (RQ3) Who removes SATD? To perform our study, we first devise a technique to determine SATD introduction and removal from the projects. In total, we examine \todo{add number} instances of SATD removals.
\begin{itemize}
	\item[\textbf{RQ}]\textbf{1:} How much self-admitted technical debt gets removed?
	\item[\textbf{RQ}]\textbf{2:} Who removes self-admitted technical debt? Is it most likely to be self-removed or removed by others?
	\item[\textbf{RQ}]\textbf{3:} RQ3. How long does self-admitted technical debt survive in a project?
	\item[\textbf{RQ}]\textbf{4:} What activities lead to the removal of self-admitted technical debt?
	
\end{itemize}

%Our findings
\textbf{Paper contribution:} Our findings indicate that 
\begin{itemize}
	\item We find that the majority of \SATD comments gets removed over the evolution of the project. Considering all projects the percentage of \SATD comments removal ranges between 40.5 - 90.6\%, and on average 74.9\% of the identified \SATD is removed.
	\item We find that most \SATD (on average 54.4\% and median 61.0\%) is self-removed.
	\item We find that the median amount of time that self-removed technical debt stays in the project is 18 days whereas non self removal technical debt is 127 days. In addition we find that technical debt removals happens more when the files containing debt are edited than when they are removed.
	\item
\end{itemize}

%Organization
The rest of the paper is organized as follows; Section \todo{complete}






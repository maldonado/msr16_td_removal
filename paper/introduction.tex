% -*- root: main.tex -*-

%overview technical debt
Technical debt was first coined by Cunningham in 1993 to refer to the phenomena of taking a shortcut to achieve short term development gain at the cost of increased maintenance effort in the future \cite{Cunningham1992WPM}. The technical debt community, organized through the managing technical debt workshop \cite{MTD2016}, has studied many aspects of technical debt, including its detection \cite{Zazworka2013EASE}, impact \cite{Zazworka2011MTD} and the appearance of technical debt in the form of code smells \cite{Fontana2012MTD}. Most recently, we developed an approach to identify technical debt from code comments, referred to as self-admitted technical debt (SATD). SATD refers to the situation where developers know that the current implementation is not optimal and write comments alerting the inadequacy of the solution. 


% What people did and what is the impact of TD. What they found.
In the last few years, an increasing amount of work has focused on SATD. In particular, our prior work focused on the detection of SATD~\cite{Potdar2014ICSME} and the classification of different types of SATD and the development of datasets to enable future studies on SATD~\cite{Maldonado2015MTD}. Other work by Bavota and Russo~\cite{Bavota2016MSR} performed an empirical study of SATD on a large number of Apache projects showed that SATD is prevalent in open source projects, is long lived and is increasing over time. A study by Wehaibi et al.~\cite{Wehaibi2016SANER} examined the impact of SATD on quality and found that SATD does not necessarily relate to more defects, however, it does make the software system more complex.

%However, much technical debt remains in the projects. Why is studying removal important
Eventhough a plethora of work argues that SATD has negative impacts, prior work also showed that some SATD remains in a project for long periods of time (up to 10 years) after its introduction. Therefore, an important question becomes ``why does some SATD remain for so long?'' and whether ``all SATD needs to be paid back?''. Examining the removal of SATD can shed light on potentially healthy patterns of debt, that may need not be paid back.

%What we do in this paper
Hence, in this paper we perform an empirical study using seven large open source software projects, where we examine phenomena relating to the removal of SATD. In particular, we examine (RQ1)how much SATD is actually removed in these projects?, (RQ2) How long SATD survives in a software project? and (RQ3) Who removes SATD? To perform our study, we first devise a technique to determine SATD introduction and removal from the projects. In total, we examine \todo{add number} instances of SATD removals.

%Our findings
Our findings indicate that 

%Organization
The rest of the paper is organized as follows; Section \todo{complete}
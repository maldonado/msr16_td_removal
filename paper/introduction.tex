% -*- root: main.tex -*-
% !TEX root = main.tex

The term technical debt was first coined by Cunningham in 1993 to refer to the phenomena of taking a shortcut to achieve short term development gain at the cost of increased maintenance effort in the future \cite{Cunningham1992WPM}. The technical debt community has studied many aspects of technical debt, including its detection \cite{Zazworka2013EASE}, impact \cite{Zazworka2011MTD} and the appearance of technical debt in the form of code smells \cite{Fontana2012MTD}. 
Most recently, the notion of self-admitted technical debt (SATD) has been introduced by Potdar and Shihab~\cite{Potdar2014ICSME}.
SATD refers to the situation where developers know that the current implementation is not optimal and write comments alerting the inadequacy of the solution.




Eventhough previous work argues that SATD has negative impact on software~\cite{Wehaibi2016SANER,kameiusingTDA2016}, it has also showed that some SATD remains in a project for long periods of time (up to 10 years) after its introduction~\cite{Potdar2014ICSME}. However, most of these prior studies did not examine the removal of SATD in depth. Examining the removal of SATD can shed light on potentially healthy patterns of debt, that may not need to be paid back.


Hence, in this paper we perform an empirical study of large open source software projects, and examine phenomena relating to the removal of SATD. In particular, we examine the following questions:
\begin{itemize}
	\item[\textbf{RQ}]\textbf{1:} \emph{How much self-admitted technical debt gets removed?} Non-removal of SATD suggests relative lack of importance of SATD for the developers. 
	\item[\textbf{RQ}]\textbf{2:} \emph{Who removes self-admitted technical debt? Is it most likely to be self-removed or removed by others?} One would expect the person that introduced SATD is better aware of the presence of SATD, and, hence, a priori, is more likely to remove SATD, i.e., to pay it back.
	\item[\textbf{RQ}]\textbf{3:} \emph{How long does self-admitted technical debt survive in a project?} Continuing the distinction between developers removing their own SATD as opposed to those removing SATD introduced by others, we would expect the former to remove SATD faster than the latter.
	\item[\textbf{RQ}]\textbf{4:} \emph{What activities lead to the removal of self-admitted technical debt?} Developers conduct both activities such as refactoring or code improvement that might explicitly target removal of technical debt, and activities related to new functionality or bug fixing that might lead to SATD removal as a byproduct.
\end{itemize}

To answer the aforementioned questions, we leverage a natural language processing (NLP) based technique, previously proposed by Maldonado \emph{et al.} in~\cite{Maldonado2015TSE}, to determine SATD introduction and removal. In total, we examine 5,733 SATD removals in five large open source projects. Our findings indicate that 1) the majority of \SATD comments are removed and in the studied projects the removal ranges between 40.5--90.6\%, and on average 74.9\% of the identified \SATD is removed; 2) most \SATD (on average 54.4\% and median 61.0\%) is self-removed; 3) the median amount of time that \SATD stays in the project ranged between 82--613.2 days on average and 18.2--172.8 days; and 4) developers add \SATD to track potential future bugs and code that needs improvements, whereas, developers mostly remove \SATD when they are fixing bugs or adding new features. Very seldom do developers remove \SATD as part of refactoring efforts or dedicated code improvement activities.

Our empirical findings provide insights to developers and software projects on how to best manage \SATD. For example, our findings show that most \SATD is self-removed, providing insight on who typically removes \SATD in large open source projects.

To ease replication, we make all the data used in our study available online~\footnote{\url{http://das.encs.concordia.ca/uploads/2017/06/maldonado_icsme2017.zip}}.

The rest of the paper is organized as follows: after detailing the case study setup in Section~\ref{sec:approach} we present the results in Section~\ref{sec:case_study_results}. We position our results with respect to the related work in Section~\ref{sec:related_work}, and evaluate threats to validity in Section~\ref{sec:threats_to_validity}.  Section~\ref{sec:conclusion} concludes and sketches future work.






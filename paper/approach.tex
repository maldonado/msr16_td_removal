% -*- root: main.tex -*-

The main goal of our study is to understand what happens with \SATD comments after they are introduced in projects. To do that, we divided our study in two main parts. First, we use a manually classified dataset that contains \SATD comments from three open source projects. Then, we trace each \SATD finding when and by who the technical debt was introduced/removed. Lastly, we run our analysis and examine the results. Second, to scale our approach to more projects we implement a process that does not depends on a manually classified dataset. Using this approach we extracted \SATD comments from other five open source projects, and similarly we analyze the \SATD comments of these projects. Figure~\ref{} shows an overview of our manual approach, Figure~\ref{} shows an overview of our automatic approach, and the following subsections detail each step.


% \subsection*{Manual Examination}
% \label{sub:manual_examination}

% \subsubsection*{Project Data Extraction}
% \label{subsub:manual_project_data_extaction}

% falar dos projetos selecionados, falar que o dataset eh oriundo de outro projeto. falar que selecionamos apenas design technical debt porque ele da melhores resultados em precisao nos vamos discutir isso depois. falar que foi manualmente classificado e agreement betwe
% criar tabela com dados dos projetos manuais.

% \subsubsection*{Extracting Different Versions of Technical Debt}

% \subsubsection*{Find Technical Debt Authors}


% \subsection*{Automatic Examination}